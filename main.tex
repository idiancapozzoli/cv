\documentclass[a4paper,10pt]{article}
\usepackage[utf8]{inputenc}
\usepackage{geometry}
\geometry{a4paper, margin=1in}
\usepackage{enumitem}
\usepackage{hyperref}
\usepackage{amsmath}

\usepackage{helvet}

\pagestyle{empty}

\begin{document}

%======================
% Name & Contact Information
%======================
\begin{center}
    {\LARGE \textbf{IDIAN CAMARGO CAPOZZOLI}} \\
    \vspace{0.2cm}
    \href{mailto:idiancapozzoli@usp.br}{idiancapozzoli@usp.br}

\end{center}

%======================
% EDUCATION
%======================
\section*{EDUCATION}
\vspace{-1.5em} % Adjust vertical space
\noindent\rule{\textwidth}{0.4pt} % Horizontal line

\noindent\textbf{University of São Paulo}, São Paulo, Brazil \hfill  \textit{Apr 2021 -- Present} \\
\textit{Bachelor's in Computer Science} \hfill GPA: 8.4/10 \\
The Computer Science program is a four-year course that offers great flexibility in selecting courses. It features a system of specialization tracks, including Theoretical Computer Science, in which I have specialized. \\

\noindent\textbf{Federal Institute of Education, Science and Technology of São Paulo} \hfill  \textit{Feb 2019 -- Dec 2019} \\
\textit{Technical Program in Informatics integrated with High School} \hfill GPA: 9.09/10 \\
The three-year course at Federal Institute of Education, Science and Technology of São Paulo -- Bragança Paulista Campus (IFSP -- BRA) trains professionals to develop and maintain software systems, perform computer maintenance, and set up local networks, with a focus on quality and security.

%\vspace{0.2cm}

%======================
% PROJECTS AND EXPERIENCE
%======================
\section*{PROJECTS AND EXPERIENCE}
\vspace{-1.5em} % Adjust vertical space
\noindent\rule{\textwidth}{0.4pt} % Horizontal line

\noindent\textbf{Undergrad Research Project}  \hfill  \textit{Sep 2023 -- Present} \\
A scientific research under the supervision of Profa. Yoshiko Wakabayashi and funded by CNPq from 2023
up to now. Initially, the project was centered on the study of various topics in combinatorics and graph theory. The current focus is the study of longest paths in graphs from a structural perspective. \\

\noindent\textbf{Undergrad Research Project}  \hfill  \textit{Oct 2021 -- Feb 2023} \\
Project entitled "Creation of an online database of archaeobotanical reference collections from Amazonia", funded by FAPESP (grant 2021/10041-2) and supervised by Jennifer Georgina Watling. The aim of the project was to create a useful tool for researchers studying Amazonian archaeobotany both in Brazil and around the world. \\ 

\noindent\textbf{Teaching Assistant}  \hfill  \textit{Feb 2024 - Jul 2024}\\
Was teaching assistant for the course Fundamentals of Mathematics for Computing.
Responsible for holding office hours and grading assignments. \\

\noindent\textbf{Computer Science Week Organizing Committee} \hfill  \textit{Oct 2023}\\
This event is organized by students from the Bachelor’s degree in Computer Science of the Institute of Mathematics and Statistics at the University of São Paulo, with support from the course's Coordinating Committee. It features lectures on various areas of Computer Science, such as Artificial Intelligence, Cryptography, Data Analysis, Game Development, Free Software, among others. \\

\noindent\textbf{CINIME Organizing Committee} \hfill  \textit{Feb 2023 - Dec 2023}\\
CINIME is the film collective of the Institute of Mathematics and Statistics at the University of São Paulo. The organizing committee is responsible for curating the films and promoting discussion sessions after the screenings. \\

\noindent\textbf{Teaching Assistant} \hfill  \textit{Feb 2019 - Dec 2019}\\
Was teaching assistant for High School Mathematics couse at  Federal Institute of Education, Science and Technology of São Paulo (IFSP). Responsible for holding office hours. \\

\noindent\textbf{Junior Scientific Initiation Program} \hfill \textit{2014 - 2017} \\
The Junior Scientific Initiation Program (PIC) provides award-winning students from the OBMEP (Brazilian Public Schools Mathematics Olympiad) with opportunities to engage with interesting issues in the field of Mathematics, enhancing their scientific knowledge and preparing them for future professional and academic performance. Students participating in the PIC receive a scholarship from CNPq during the program year. The student participated in the editions of 2014, 2015, 2016, and 2017.


\vspace{0.2cm}
%======================
% PARTICIPATION IN EVENTS
%======================
\section*{PARTICIPATION IN EVENTS}
\vspace{-1.5em} % Adjust vertical space
\noindent\rule{\textwidth}{0.4pt} % Horizontal line

\noindent\textbf{6th São Paulo Workshop on Optimization, Combinatorics, and Algorithms} \hfill \textit{9 - 13 Oct 2024}\\
The São Paulo Workshop on Optimization, Combinatorics, and Algorithms (WoPOCA) is an event that aims to create a collaborative environment for discussing open problems in optimization, combinatorics, and algorithms, involving professors, graduate, and undergraduate students. Worked mainly in the Balanced Connected \(k\)-Partition Problem for planar graphs. The student also participates in the mini-course 'Matroids and Greedy Algorithms' ministred by Prof. Orlando Lee. \\

\noindent\textbf{International Symposium on Scientific and Technological Initiation of USP} \hfill \textit{22, 23 Oct 2024}\\
The International Symposium on Scientific and Technological Initiation of the University of São Paulo (SIICUSP) is an annual event that highlights research findings from undergraduate students at USP. The best undergraduate research projects are selected to an international Symposium from University of São Paulo.The project "Longest Paths in Graphs: Structural and Algorithmic Approach" was present at the 32º SIICUSP, under the supervision of Profa. Yoshiko Wakabayashi. \\

\noindent\textbf{Congress on Innovation, Science, and Technology of IFSP} \hfill \textit{Sep 2019}\\
Conict - Congress on Innovation, Science, and Technology is organized by the Research and Graduate Studies Office (PRP) of the Federal Institute of Education, Science, and Technology of São Paulo (IFSP). It is an event open to the participation of high school and higher education students who are conducting research at IFSP or other educational or research institutions in the country.The project "Use of Drones for Image Processing and Calculation of Polygonal Target Areas" was presented and also published in the proceedings of Conict. \\

\noindent\textbf{BRAGANTEC - Regional Science and Technology Fair} \hfill \textit{Sep 2018 and Sep 2019}\\
BRAGANTEC is an annual Science and Technology Fair in the Bragança region, organized by IFSP - Bragança Paulista Campus. The project "MEILEARN: website and platform for study" was presented in 2018 under the supervision of Ana Paula Muller Giancoli. And the project "GEOMATIC: software for geometry learning" was presented in 2019 under the supervision of Diana Terezinha Amaro and Lucas Miguel Carvalho. This project was also presented in other regional event "Science Fair and Technological Exhibition and Entrepreneurship of Salto – IFCIÊNCIA", organized by IFSP – Salto Campus.
%


%======================
%HIGH SCHOOL ACADEMIC ACHIEVEMENTS
%======================
\section*{ACADEMIC ACHIEVEMENTS}
\vspace{-1.5em} % Adjust vertical space
\noindent\rule{\textwidth}{0.4pt} % Horizontal line
\begin{itemize} 
    \item Silver Medal in the Brazilian Public Schools Mathematics Olympiad (2019).
    \item Gold Medal in the Mathématiques sans Frontières (2018).
    \item Silver Medal in the São Paulo Mathematics Olympiad (2018).
    \item Honorable Mention in the Brazilian Public Schools Mathematics Olympiad (2018).
    \item Silver Medal in the Mathématiques sans Frontières (2017).
    \item Honorable Mention in the Brazilian Public Schools Mathematics Olympiad (2017).
    \item Bronze Medal in the Brazilian Public Schools Mathematics Olympiad (2016).
    \item Bronze Medal in the Brazilian Astronomy and Astronautics Olympiad (2016).
    \item Bronze Medal in the Brazilian Public Schools Mathematics Olympiad (2015).
    \item Bronze Medal in the Brazilian Astronomy and Astronautics Olympiad (2015).
    \item Bronze Medal in the Brazilian Public Schools Mathematics Olympiad (2014).
    \item Bronze Medal in the Brazilian Astronomy and Astronautics Olympiad (2014).
    \item Bronze Medal in the Brazilian Public Schools Mathematics Olympiad (2013).
\end{itemize}


\end{document}
